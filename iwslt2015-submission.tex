\documentclass[a4paper]{article}
\usepackage{iwslt15,amssymb,amsmath,epsfig}
\usepackage{verbatim}
\setcounter{page}{1}
\sloppy		% better line breaks
%\ninept
%SM below a registered trademark definition
\def\reg{{\rm\ooalign{\hfil
     \raise.07ex\hbox{\scriptsize R}\hfil\crcr\mathhexbox20D}}}

%% \newcommand{\reg}{\textsuperscript{\textcircled{\textsc r}}}

\title{Integrating Semantic information into Neural Network Language Models}

%%%%%%%%%%%%%%%%%%%%%%%%%%%%%%%%%%%%%%%%%%%%%%%%%%%%%%%%%%%%%%%%%%%%%%%%%%
%% Please make sure to keep technical paper submissions anonymous  !
%%%%%%%%%%%%%%%%%%%%%%%%%%%%%%%%%%%%%%%%%%%%%%%%%%%%%%%%%%%%%%%%%%%%%%%%%%
\name{Firstname Lastname}
%%%%%%%%%%%%%%%%%%%%%%%%%%%%%%%%%%%%%%%%%%%%%%%%%%%%%%%%%%%%%%%%%%%%%%%%%%
%% If multiple authors, uncomment and edit the lines shown below.       %%
%% Note that each line must be emphasized {\em } by itself.             %%
%% (by Stephen Martucci, author of spconf.sty).                         %%
%%%%%%%%%%%%%%%%%%%%%%%%%%%%%%%%%%%%%%%%%%%%%%%%%%%%%%%%%%%%%%%%%%%%%%%%%%
% \makeatletter
% \def\name#1{\gdef\@name{#1\\}}
% \makeatother
% \name{{\em Firstname1 Lastname1, Firstname2 Lastname2, Firstname3 Lastname3,}\\
%      {\em Firstname4 Lastname4}}
%%%%%%%%%%%%%%% End of required multiple authors changes %%%%%%%%%%%%%%%%%

\address{Insitute for Anthropomatics  \\
Karlsruhe Institute of Technology, Germany \\
{\small \tt firstname.lastname@iwslt.org}
}
%
\begin{document}
\maketitle
%
\begin{abstract}
Neural models have recently shown big improvements in the performance of low-resource language modeling in phrase-based machine translation. Recurrent language models with different word factors, in particular, were a great success due to their ability to incorporate additional knowledge into the model. In this work, we want to integrate global semantic information extracted from large independent knowledge bases into neural network language models. We propose two approaches for doing this: word class extraction from Wikipedia and sentence level topic modeling. 
The new resulting models exhibit great potential in counteracting data scarcity problems with additional independent knowledge. This approach of integrating global context information is not restricted to language modeling but can also be easily applied to any model that profits from context or further data resources, e.g. neural machine translation. Using this model has improved rescoring quality of a state-of-the-art phrase-based translation system by ... BLEU points.  We performed experiments on two language pairs.



\end{abstract}


%
\section{Introduction}
Recurrent neural network language models have recently shown great improvement in statistical machine translation, both during decoding and rescoring. The use of continuous word representations has achieved better generalizations of the data which effectively lowered data sparseness problems. Furthermore, the recurrent connections are able to model long range dependencies. Yet, most of these models strictly depend on monolingual and parallel data, which is sometimes not available in huge amounts, especially for low-resource languages.
This has motivated neural network language models that take multiple parallel streams of data as input instead of just the single form of surface words. These so called factors can be used to add additional information, e.g. POS or automatic word clusters, which helps mainly with morphologically rich languages (e.g. Romanian, German). However, so far the use of factors or additional information has been limited in neural network models. Also, in those cases the extra feature only pertained to syntactic or local context knowledge around the current word. Especially for languages with low resources, it is essential to also facilitate the use of other knowledge factors, e.g. encyclopedia knowledge. It is a useful source especially for learning general concepts, even more after the emergence of the Internet has led to an explosion of textual data. These data sources give insights into a variety of human endeavors waiting to be computationally analyzed.
In this paper, we study the integration of large independent knowledge bases in the form of encyclopedia, e.g. Wikipedia, into RNN-based language models and propose two solutions. 
First, we use  the factored model to integrate extracted Wikipedia categories as one of the factors. In order to understand large unstructured datasets great achievements have been attained in latent concept learning in the area of text mining. Techniques include categorization of documents using latent semantic analysis and probabilistic topic modeling. Therefore, we employed also these techniques to compute a real-valued topic vector for each sentence that is fed into the network as additional input. Using word classes and semantic features help both sparsely inflected languages(e.g. English, Chinese) \cite{bilmes2003factored} as well as low-resource languages.
In the model we use LSTMs to take into account both independent side information and local context information for the model prediction.


\section{Related Work}
Language models are a critical component of many application systems, e.g. ASR, MT and OCR. However, language models have always faced the problem of data sparseness. Factored Language Models \cite{bilmes2003factored} introduced the use of a bundle of factors associated with a word which outperformed previous n-gram models without expanding the training data. For factors morphs, stems, POS and word class obtained using the SRILM's n-gram-class tool were used. \cite{koehn2007factored} replaced the single feature stream of surface words with multiple factors and integrated it into phrase-based statistical machine translation systems by breaking down the translation model into several steps that pertain to the translation of single factors which are all taken into account when the target word is generated.
After recurrent neural network models became a success in language modeling \cite{mikolov2010recurrent}, a factored input layer was employed in a model by \cite{wu2012factored} which uses a structured output layer based on word classes that was able to handle vocabulary of arbitrary size.
Motivated by multi-task learning in NLP, \cite{niehuesusing} proposed a multi-factor recurrent neural network language model which jointly predicts different output factors by mapping the output of the LSTM-layer to as many softmax layers as there are output factors, thus creating multiple distributions at the output layer. In the rescoring of an n-best list, this model can be included as either one additional feature or several features depending on whether the output is treated as a joint probability or individual probabilities.
One disadvantage of factored input is that additional factors must match the surface words in space. As a consequence, surface words that uses 1-of-n encoding cannot have factors with continuous space representations.
However, this is often necessary to model more complex structures, e.g. topic distributions. 
In \cite{mikolov2012context}, a topic-conditioned RNNLM is proposed which takes a real-valued input vector as an additional input in association with each word. This vector is used to convey local context information based on previous sentences using latent dirichlet allocation.
Often, the meaning of a word cannot be just derived from its preceding words but by content words in the entire sentence or surrounding sentences. 
However, the model cannot take side information associated with a sentence that contains the current word, which is what we studied in the second part.

\section{Integration of Side Information}

\subsection{Word class extraction}
\begin{comment}
The paper title must be in boldface. All non-function words must be capitalized,
and all other words in the title must be lower case. The paper title is centered
across the top of the two columns on the first page as indicated above.
\end{comment}


\subsection{Sentence level topic modeling}

\begin{comment}
The authors' name(s) and affiliation(s) appear centered below the paper
title. If space permits, include a mailing address here. The templates indicate
the area where the title and author information should go. These items need not
be confined to the number of lines indicated; papers with multiple
authors and affiliations may require two or more lines. 
Note that the submission version of technical papers \emph{should be 
anonymized for review}. 
\end{comment}




\section{Experiments}
\subsection{System Description}
\subsection{English-Chinese}
\subsection{English-Romanian}
\section{Conclusion}
\section{Acknowledgements}

\begin{comment}

The IWSLT 2015 organizing committee would like to thank the
organizing committees of INTERSPEECH 2004 for their
help and for kindly providing the template files.

\begin{itemize}
%\itemsep -1.3mm
\item Proceedings will be printed in A4 format. The layout is designed 
so that files, when printed in US Letter format, include all material 
but margins are not symmetric. 
Although this is not an absolute requirement, if at all possible,
{\bf PLEASE TRY TO MAKE YOUR SUBMISSION IN A4 FORMAT.}
\item Two columns are used except for the title part and possibly for large 
figures that need a full page width.
\item Left margin is 20 mm.
\item Column width is 80 mm.
\item Spacing between columns is 10 mm.
\item Top margin 25 mm (except first page 30 mm to title top).
\item Text height (without headers and footers) is maximum 235 mm.
\item Headers and footers must be left empty (they will be added for 
printing).
\item Check indentations and spacings by comparing to this 
example file (in pdf format).
\end{itemize}



\subsubsection{Headings}

Section headings are centered in boldface
with the first word capitalized and the rest of the heading in 
lower case. Sub-headings appear like major headings, except they 
start at the left margin in the column.
Sub-sub-headings appear like sub-headings, except they are in italics 
and not boldface. See the examples given in this 
file. No more than 3 levels of headings should be used.

\subsection{Text font}

Times or Times Roman font is used for the main text. Recommended 
font size is 9 points which is also the minimum allowed size.
Other font types may be used if needed for 
special purposes. While making the final PostScript file, 
remember to include all fonts!

\LaTeX\ users: DO NOT USE Computer Modern FONT FOR TEXT (Times is 
specified in the style file). If possible, make the final 
document using POSTSCRIPT FONTS.
This is necessary given that, for example, equations with 
non-ps Computer Modern are very hard to read on screen.

\subsection{Figures}

All figures must be centered on the column (or page, if the figure spans 
both columns).
Figure captions should follow each figure and have the format given in 
Fig.~\ref{spprod}.

Figures should preferably be line drawings. If they contain gray 
levels or colors, they should be checked to print well on a 
high-quality non-color laser printer.

\subsection{Tables}

An example of a table is shown as Table \ref{table1}. Somewhat 
different styles are allowed according to the type and purpose of the 
table. The caption text may be above or below the table.

\begin{table}
\caption{\label{table1} {\it This is an example of a table.}}
\vspace{2mm}
\centerline{
\begin{tabular}{|c|c|}
\hline
ratio & decibels \\
\hline  \hline
1/1 & 0 \\
2/1 & $\approx 6$ \\
3.16 & 10 \\
10/1 & 20 \\ 
1/10 & -20 \\
\hline
\end{tabular}}
\end{table}

\subsection{Equations}

Equations should be placed on separate lines and numbered. Examples 
of equations are given below.
Particularly,
%
%\vspace{-3mm}
\begin{equation}
x(t) = s(f_\omega(t))
\label{eq1}
\end{equation}
where \(f_\omega(t)\) is a special warping function
\begin{equation}
f_\omega(t)=\frac{1}{2\pi j}\oint_C \frac{\nu^{-1k}d\nu}
{(1-\beta\nu^{-1})(\nu^{-1}-\beta)}
\label{eq2}
\end{equation}
A residue theorem states that
\begin{equation}
\oint_C F(z)dz=2 \pi j \sum_k Res[F(z),p_k]
\label{eq3}
\end{equation}
Applying (\ref{eq3}) to (\ref{eq1}), 
it is straightforward to see that
\begin{equation}
1 + 1 = \pi
\label{eq4}
\end{equation}

Make sure to use \verb!\eqref! when refering to equation numbers.
Finally we have proven the secret theorem of all speech sciences (see
equation~\eqref{eq3} above).  No more math is needed to show how 
useful the result is! 

\begin{figure}[t]
\centerline{\epsfig{figure=figure,width=40mm}}
\caption{{\it Schematic diagram of speech production.}}  
\label{spprod}
\end{figure}

\subsection{Hyperlinks}

Hyperlinks can be included in your paper. Moreover, be aware that the paper
submission procedure includes the option of specifying a hyperlink for
additional information.  This hyperlink will be included in the CD-ROM.
Particularly pay attention to the possibility, from this single hyperlink, to
have further links to information such as other related documents, sound or
multimedia.

If you choose to use active hyperlinks in your paper, 
please make sure that they present no problems in printing to paper. 

\subsection{Page numbering}

Final page numbers will be added later to the document
electronically. 
{\em Please don't make any headers or footers!}.

\subsection{References}

The reference format is the standard for IEEE publications.
References should be numbered in order of appearance, 
for example \cite{ES1}, \cite{ES2}, and \cite{ES3}. 

\section{Experiments}
Please make sure to give all the necessary details regarding your experimental 
setting so as to ensure that your results could be reproduced by other teams. 

\section{Conclusions}

This paper has described a novel approach for doing wonderful stuff such as ...
content...
\end{comment}

%
\bibliographystyle{IEEEtran}
\bibliography{references}
%\begin{comment}
%\bibitem[1]{ES1} Smith, J. O. and Abel, J. S., 
%``Bark and {ERB} Bilinear Transforms'', 
%IEEE Trans. Speech and Audio Proc., 7(6):697--708, 1999.  
%\bibitem[2]{ES2} Lee, K.-F., Automatic Speech Recognition: 
%The Development of the 
%SPHINX SYSTEM, Kluwer Academic Publishers, Boston, 1989.
%\bibitem[3]{ES3} Rudnicky, A. I., Polifroni, Thayer, E. H.,
% and Brennan, R. A.  
%"Interactive problem solving with speech", J. Acoust. Soc. Amer., 
%Vol. 84, 1988, p S213(A).
%\end{comment}
\end{document}

